% Präambel
\documentclass[12pt,a4paper,oneside, 
liststotoc, 					% Tabellen- und Abbildungsverzeichnis ins Inhaltsverzeichnis
bibtotoc,						% Literaturverzeichnis ins Inhaltsverzeichnis aufnehmen
titlepage, 						% Titlepage-Umgebung statt \maketitle
headsepline, 					% horizontale Linie unter Kolumnentitel
%abstracton,					% Überschrift beim Abstract einschalten, Abstract muss dazu in {abstract}-Umgebung stehen
%DIV11,							% auskommentieren, um den Seitenspiegel zu vergrößern
BCOR6mm,						% Bindekorrektur, die den Seitenspiegel um 6mm nach rechts verschiebt,
]{scrreprt}							% Dokument in utf8-Codierung schreiben und speichern
\usepackage[utf8]{inputenc} 	% ermöglicht die direkte Eingabe von Umlauten
\usepackage[ngerman]{babel} 	% deutsche Trennungsregeln und Übersetzung der festcodierten Überschriften
\usepackage[T1]{fontenc} 		% Ausgabe aller zeichen in einer T1-Codierung (wichtig für die Ausgabe von Umlauten!)		
\usepackage{textcomp} 			% zum Einsatz von Eurozeichen u. a. Symbolen
\usepackage{listings}			% Darstellung von Quellcode mit den Umgebungen {lstlisting}, \lstinline und \lstinputlisting
\usepackage{xcolor} 			% einfache Verwendung von Farben in nahezu allen Farbmodellen
\usepackage[intoc]{nomencl} 	% zur Erstellung des Abkürzungsberzeichnisses
\usepackage{fancyhdr}			% Zusatzpaket zur Gestaltung von Fuß und Kopfzeilen
\usepackage{graphicx}
\usepackage{hyperref}
\usepackage[ngerman]{cleveref} 
\usepackage[backend=biber,style=authoryear]{biblatex}
\usepackage{csquotes}
\addbibresource{Studienarbeit.bib}


% -----------------------------------------------------------------------------------------------------------------
% Zum Aktualisieren des Abkürzungsverzeichnisses bitte auf der Kommandozeile folgenden Befehl aufrufen :
%  makeindex Studienarbeit.nlo -s nomencl.ist -o Studienarbeit.nls
% -----------------------------------------------------------------------------------------------------------------

% Hier die persönlichen Daten eingeben:

\newcommand{\titel}{BYOD}
\newcommand{\untertitel}{Konzeptionierung einer Entscheidungsempfehlung für ein mittelständiges Unternehmen}
\newcommand{\arbeit}{Studienarbeit}
\newcommand{\studiengang}{Informationstechnik}
\newcommand{\autor}{Nicolas Konle, Luka Kröger}
\newcommand{\matrikelnr}{MATRIKELNUMMERN}
\newcommand{\kurs}{TINF15B3}
\newcommand{\abgabe}{\today}
\newcommand{\betreuerdhbw}{Ralf Brune}

\newcommand{\jahr}{2018}			% für Angabe im Copyright-Vermerk der Titelseite

% Abkürzungen
\newcommand{\ua}{\mbox{u.\,a.\ }}
\newcommand{\zB}{\mbox{z.\,B.\ }}
\newcommand{\bs}{$\backslash$}

\renewcommand{\nomname}{Abkürzungsverzeichnis}

% -------------------------------------------------------------------------------------------
% Definition der Kopf- und Fußzeilen
\lhead{}								% Kopf links
\chead{}								% Kopf mitte
\rhead{\sffamily{\titel}}				% Kopf rechts
\lfoot{}								% Fuß links
\cfoot{\sffamily{\thepage}}				% Fuß mitte
\rfoot{\sffamily{\autor}}				% Fuß rechts
\renewcommand{\headrulewidth}{0.4pt}	% Liniendicke Kopf
\renewcommand{\footrulewidth}{0.4pt}	% Liniendicke Fuß


\makenomenclature							% Abkürzungsverzeichnis erstellen

% alle Abkürzungen, die in der Studienarbeit verwendet werden

\nomenclature{DHBW}{Duale Hochschule Baden-Württemberg}
\nomenclature{Sem}{Semester}
\nomenclature{OSS}{Open Source Software}					% Datei mit Abkürzungen laden

% -------------------------------------------------------------------------------------------
%                     Beginn des Dokumenteninhalts
% -------------------------------------------------------------------------------------------
\begin{document}
\setcounter{secnumdepth}{3}					% Nummerierungstiefe fürs Inhaltsverzeichnis
\setcounter{tocdepth}{3}
\sffamily									% für die Titelei serifenlose Schrift verwenden

% ------------------------------ Titelei -----------------------------------------------------

\thispagestyle{plain}
\begin{titlepage}
\enlargethispage{4.0cm}
\sffamily 								% Serifenlose Grundschrift für die Titelseite einstellen
				
\begin{flushright}
\includegraphics[scale=2.0]{Bilder/logo_dhbw.jpg}\\[5ex]
\end{flushright}

\begin{center}

\huge{\textsc{\textbf{\titel}}}\\[1.5ex]
\Large{\textbf{\untertitel}}\\[5ex]
\LARGE{\textbf{\arbeit}}\\[2ex]
\normalsize{für die Prüfung zum\\[1ex] Bachelor of Engineering}\\[3ex]
\Large{Studiengang \studiengang}\\[1ex]
\normalsize{Duale Hochschule Baden-Württemberg Karlsruhe}\\[5ex]
von\\[1ex] \autor \\[18ex]


\end{center}

\begin{flushleft}

\begin{tabular}{ll}
Abgabedatum:					& \quad \abgabe \\
Bearbeitungszeitraum:			& \quad 12 Wochen   \\ 
Matrikelnummer, Kurs: 			& \quad \matrikelnr , \kurs \\ 
Betreuer der Dualen Hochschule: & \quad \betreuerdhbw \\ [5ex]

\end{tabular} 



\small
Copyrightvermerk:\\

Dieses Werk einschließlich seiner Teile ist \textbf{urheberrechtlich geschützt}. Jede Verwertung außerhalb der engen Grenzen des Urheberrechtgesetzes ist ohne Zustimmung des Autors unzulässig und strafbar. Das gilt insbesondere für Vervielfältigungen, Übersetzungen, Mikroverfilmungen sowie die Einspeicherung und Verarbeitung in elektronischen Systemen.
\end{flushleft}
\begin{flushright}
\copyright{} \jahr
\end{flushright}
\end{titlepage} 				% erzeugt die Titelseite
\pagenumbering{Roman}						% große, römische Seitenzahlen für Titelei
\addchap{Eidesstattliche Erklärung}
Ich versichere hiermit, dass ich meine Studienarbeit mit dem Thema
\begin{quote}
\textit{\titel} -\textit{ \untertitel }
\end{quote}
selbständig verfasst und keine anderen als die angegebenen Quellen und Hilfsmittel benutzt habe. Die Arbeit wurde bisher keiner anderen Prüfungsbehörde vorgelegt und auch nicht veröffentlicht.


Mir ist bekannt, dass ich meine Diplomarbeit zusammen mit dieser Erklärung fristgemäß nach Vergabe des Themas in dreifacher Ausfertigung und gebunden im Sekretariat meines Studiengangs an der DHBW Karlsruhe abzugeben habe. Als Abgabetermin giltbei postalischer Übersendung der Eingangsstempel der DHBW, also nicht der Poststempel oder der Zeitpunkt eines
Einwurfs in einen Briefkasten der DHBW.\\[10ex]

Karlsruhe, den \today \\[4ex]


\rule[-0.2cm]{5cm}{0.5pt} \\

\textsc{\autor} \\[10ex]

% Sperrvermerk bei Bedarf dekommentieren
\hrule 
\vspace*{1.0cm}
\noindent \textbf{\Large{Sperrvermerk}}\\
\normalsize 				% Einbinden der eidestattlichen Erklärung
\chapter*{Abstract/Zusammenfassung} %*-Variante sorgt dafür, das Abstract nicht im Inhaltsverzeichnis auftaucht

Hier bitte den Abstract Ihrer Arbeit eintragen. Der Abstract sollte nicht länger als eine halbe Seite sein. Bitte klären Sie mit Ihrem Studiengangsleiter ab, ob der Abstract in englischer oder deutscher Sprache (oder möglicherweise sogar in beiden Sprachen) verfasst werden soll.
   				% Einbinden des Abstracts

\tableofcontents							% Erzeugen des Inhalsverzeichnisses
\printnomenclature[2.0cm]					% Erzeugen des Abkürzungsverzeichnisses
\listoffigures 								% Erzeugen des Abbildungsverzeichnisses 
\listoftables 								% Erzeugen des Tabellenverzeichnisses
\pagebreak

% --------------------------------------------------------------------------------------------
%                    Inhalt der Studienarbeit
%---------------------------------------------------------------------------------------------
\pagenumbering{arabic}						% arabische Seitenzahlen für den Hauptteil
\pagestyle{fancy}					
\rmfamily

\chapter{Einleitung}
\label{cha:Einleitung}

\section{Motivation}
\label{sec:Motivation}

\section{Ziel der Arbeit}
\label{sec:ZielDerArbeit}

\section{Aufbau der Arbeit}
\label{sec:AufbauDerArbeit}

\section{Bring Your Own Device}
In der heutigen Arbeitswelt hat die Geschwindigkeit ein Produkt auf den Markt zu bringen so sehr an Wert gewonnen, dass man dem ständigen Wettbewerb gerecht wird und als Unternehmen überlebt. Hierbei ist es nicht nur wichtig schnell zu sein, sondern auch flexibel. Dazu gehört in einem Unternehmen, die Möglichkeit, wo- und wann auch immer, Arbeit zu verrichten, zu den Voraussetzungen und Treibern, Erfolg im Wettkampf mit der Konkurrenz zu sichern. Die Denkweise der Arbeitnehmer heutiger Zeit, hat sich durch Globalisierung und Vernetzung so angepasst, dass es darum geht, welche Arbeit getan werden muss, und nicht wo. Das Equipment, dass von den Unternehmen den Arbeitnehmern aufgezwungen wird ist nicht nutzerfreundlich und veraltet, der Arbeitsplatz soll nicht mehr auf das Büro beschränkt sein und ein Arbeiten von zuhause und unterwegs ist wegen der ständigen Bewegung in der "Online"-Welt nicht mehr wegzudenken. Die Grenze zwischen Arbeit und Leben der Arbeitnehmer vermischt sich immer mehr.

Bring Your Own Device (BYOD) beschreibt eine IT-Richtlinie, die, wie die deutsche Übersetzung "Bring dein eigenes Gerät" offenbart, Mitarbeitern eines Unternehmens erlaubt, eigene private Geräte im geschäftlichen Umfeld zu nutzen. Zu den Geräten dieser \"Trend\"-Richtlinie gehören hauptsächlich die Nutzung von Smartphones. Tablets oder Laptops sind ebenfalls umsetzbar, aber werden im Rahmen dieser Arbeit nicht behandelt. Um BYOD umsetzen zu können, werden EMM-Plattformen eingesetzt (siehe \cref{sec:EMM}), welche die Trennung einer Arbeits- und Privatwelt auf den Geräten ermöglicht und dabei die Sicherheitsrichtlinien des Unternehmens einhält. Bevor jedoch erklärt wird, wie BYOD Lösungen umgesetzt werden können, gilt zuerst die Frage zu beantworten warum und wie es zu dieser IT-Bewegung kommt.

\subsection{Vorteile}
In einer Studie von Extreme Networks \cite{ext2014} wurden deutsche Unternehmen nach Vorteilen, die sie sich aus dem Nutzen von BYOD erhoffen. Als Ergebnis, siehe \cref{fig:VorBYOD} sind 43\% erhöhte Mitarbeiterzufriedenheit, 32\% erhöhte Produktivität der Mitarbeiter, 16\% Kosteneinsparungen und 9\% andere Gründe zu vermerken.
\begin{figure}[hbt]
\centering
\includegraphics[width=0.95\textwidth]{Bilder/Vorteile_BYOD.png} 
\caption{Umfrage: Erhoffte Vorteile von BYOD}\label{fig:VorBYOD}
\end{figure}
Diese Ziele lassen sich in Vorteile für den Endnutzer und für das Unternehmen unterteilen.\footnote{Vgl. \cite{fuj2018} } Dem Endnutzer bietet die Umsetzung von BYOD den Vorteil, die Freiheit, das Gerät nach ihren Präferenzen auszuwählen. Wie oben beschrieben löst BYOD das Problem der Stationärität des Arbeitnehmers. Das Arbeiten von zuhause und unterwegs ist damit kein Problem mehr. Subjektiv ist, ob BYOD einen positiven Einfluss auf das Work-Life-Balance hat, aber es hat zumindest den Vorteil, dass die Integration von Leben und Arbeit möglich ist. Daraus erhoffen sich die Unternehmen, dass durch das Nutzen eines Gerätes für das private Leben sowie für die Arbeit, die Mitarbeiterzufriedenheit und gleichzeitig das Engagement steigt.

Es ist zu erkennen, dass nicht nur dem Endnutzer aus diesem Aspekt, ein Vorteil geboten ist. BYOD, daraus abgeleitet Mitarbeiterzufriedenheit, hat als weitere Folge, die Attraktivität und das Image eines Unternehmens aufzubessern und so weitere qualifizierte Mitarbeiter zu gewinnen. Denn als Ziel von Unternehmen gilt wie in traditionellerweise die Suche von Talenten. Der Technologiefortschritt und die Flexibilität eines Arbeitsplatzes ist im Vergleich zum konventionellen Angebot eines Firmenwagens mittlerweile deutlich attraktiver und erhält immer größere Signifikanz. 

Das Mitarbeiter, ihre Arbeit mit dem Privatleben vermischen, bringt Unternehmen weitere Vorteile. Es ist Möglich zusätzlich neben den normalen Arbeitszeiten an Wochenenden zu arbeiten. Dies fördert nicht nur die komplette Produktivität sondern auch die Empfänglichkeit gegenüber der Kunden, welches weiterhin zu erhöhter Kundenzufriedenheit führt. Weiterhin verringert sich dadurch die Antwortzeit auf den Markt, fördert Innovation und sichert damit einen Vorsprung im Wettbewerb.

Wie in der Umfrage beschrieben, erhoffen sich Unternehmen einen Vorteil durch Kosteneinsparungen durch die Einführung von BYOD. Die Frage ob durch die Umsetzung ein Kostenvorteil entsteht und ist nicht direkt ablesbar, da es komplett abhängig von der vorher genutzten Infrastruktur eines Unternehmens ist. Nimmt man ein Unternehmen, welches viel Geld in Firmengelände investiert, damit dort die Mitarbeiter arbeiten können, ist ein großer Kostenvorteil denkbar. Denn lässt dieses mithilfe von BYOD einen Großteil der Mitarbeiter von zuhause arbeiten, können Ausgaben für Gelände und Arbeitsorte gespart werden. 

Da Nutzer ihr eigenes Gerät verwenden, steigt die Fürsorge und Vorsicht darum. So können Kosten für Wartung und Ersatzvergabe präventiv klein gehalten werden. Da sich Nutzer mit den Geräten sich ständig auseinandersetzen, steigt die Erfahrung und das Wissen in der Benutzung. Folglich können damit Service Dienste intern gesenkt werden und damit weiter gespart werden.

\subsection{Herausforderungen}

\chapter{Ausgangssituation}
\label{cha:Ausgangssituation}

\section{Ausgangssituation}



\chapter{Grundlagen}
\label{cha:Grundlagen}
In diesem Kapitel werden Grundlagen zum Thema BYOD gegeben die zum Verständnis der Arbeit benötigt werden. 


\section{Mobile Device Management (MDM)}
Mobile Device Management (MDM) ist eine Technologie zur Lebenszyklusverwaltung, mit der die IT Mobilgeräte durch auf den Geräten installierte MDM-Profile einsetzen, konfigurieren, verwalten, unterstützen und sichern kann. MDM Software ermöglicht Anlageninventur, Over-the-Air-Konfiguration von E-Mail, Anwendungen und WLAN, Remotefehlerbehebung sowie Remotesperr- und Remote Wipe-Funktionen zur Sicherung von Geräten und den darauf befindlichen Unternehmensdaten. MDM ist die Grundlage für eine umfassende Enterprise Mobility Management-Lösung (EMM). 

\section{Mobile Application Management}
Mobile Application Management-Technologien (MAM) wenden Tools der Verwaltungs- und Richtlinienkontrolle auf individuelle Anwendungen statt auf das gesamte Gerät an. Üblicherweise bieten MAM-Lösungen einen benutzerdefinierten App Store, der die Kontrolle und Bereitstellung von sowohl intern entwickelten als auch Drittanbieteranwendungen erlaubt. IT-Administratoren haben die Möglichkeit, mithilfe von AppConfig Community-Standards oder Software Development Kit- oder App Wrapping-Lösungen vom MAM-Anbieter der Anwendung Sicherheits-, Verschlüsselungs- und Kontrollfunktionen hinzuzufügen. 

\section{Mobile Content Management}

\section{Enterprise Mobility Management}
\label{sec:EMM}
Enterprise Mobility Management (EMM) ist eine geräte- und plattformagnostische Lösung, in der die Verwaltung, Konfiguration und Sicherheit aller Geräte – BYO sowie unternehmenseigenen – einer Organisation zusammengefasst werden. EMM erstreckt sich über traditionelle Geräteverwaltung hinaus auf die Verwaltung und Konfiguration von Unternehmensanwendungen und -inhalten.  Eine solide EMM-Lösung umfasst üblicherweise MDM, MAM, Mobile Content Management (MCM), Identitätsmanagement zur Zugriffskontrolle sowie Produktivitätsanwendungen zum mühelosen Zugriff auf E-Mail, Kalender, Kontakte, Inhalts-Repositorys und Intranet-Sites des Unternehmens. Darüber hinaus sollte eine EMM-Lösung es technisch ermöglichen, der IT die Verwaltung und Sicherheit zu erleichtern und gleichzeitig den Mitarbeitern ein angenehmes Benutzererlebnis zu bieten. 

\section{Unified Endpoint Management}
Mithilfe von Unified Endpoint Management (UEM) ist die IT endlich in der Lage, sich der unterschiedlichen Tools für die Verwaltung von Mobilgeräten, Desktops und seit Kurzem auch IdD-Geräten (Internet der Dinge) zu entledigen. Durch die Kombination von traditioneller Client-Verwaltung von Desktop- und PC-Systemen mit einem modernen Enterprise Mobility Management Framework (EMM) schaffen UEM-Lösungen die Voraussetzung für einen ganzheitlichen und anwenderorientierten Ansatz zur Verwaltung aller Endpunkte. Eine solide UEM-Lösung befähigt die IT zur Verwaltung von Benutzern und zur Realisierung eines einheitlichen Erlebnisses entlang aller Endpunkte sowie zur Sicherung und Verwaltung des gesamten Gerätelebenszyklus – und alles über eine zentrale, umfassende Plattform. 







\chapter{Datenschutz}
Durch den Einsatz privater Endgeräte im Unternehmensumfeld kann es zu wechselseitigen Zugriffsmöglichkeiten auf private, personenbezogene Daten oder besonders schützenswerte Unternehmensdaten kommen. 
Um die Interessen natürlicher Personen zu wahren existieren Rechtsvorschriften. Für dem Schutz von Unternehmensdaten hingegen müssen zusätzlich Maßnahmen zur Absicherung ergriffen werden.


\section {Rechtsrahmen}
Vor der Einführung einer Bring Your Own Device Lösung in einem Unternhemen entsteht juristischer Klärungsbedarf bezüglich der Gültigkeit verschiedener Rechtsvorschriften. 
\subsection{Bundesdatenschutzgesetz}
Das Bundesdatenschutzgesetz schreibt vor, dass 

\subsection{Europäische Datenschutz Grundverordnung}
\subsection{Arbeitsrecht}
Betriebsvereinbarungen, Vereinbarungen im Arbeitsvertrag. 
\subsection{Handelsgesetz}
\subsection{Strafrecht}
\subsection{Steuerrecht}
\section {Schützenswerte Daten}

Als schützenswerte Daten werden alle Unternehmensinternen Informationen bezeichnet, die für 



\section {Weitere Aspekte}

\subsection{Haftung}














\chapter{Systeme}
\label{cha:systeme}

\section{Samsung Knox}

Als vorinstallierte Standardsoftware aktueller Samsung-Geräten findet man die App \textit{MyKnox}. Hiermit kann ein Benutzer mit einem einzelnen Tippen auf die Applikation zwischen gesichertem und normalem Modus wechseln. In diesem gesicherten Modus ist es durch eine Containerlösung möglich, Aktivitäten, geschäftlich oder privat, durch ein Sicherheitsverfahren zu schützen. Dieses Sicherheitsverfahren besteht bei der Knox-Plattform aus fünf Komponenten \cite{sam2017}:
\begin{enumerate}
\item Mehrschichtige Sicherheit
\item Root-of-Trust
\item Secure Boot und Trusted Boot
\item TrustZone®
\item SE for Android
\end{enumerate}


  Diese Knox-Plattform soll im Folgenden nach dem Kriterienkatalog betrachtet werden.
S
Samsung bietet je nach Sicherheitsanforderung verschiedene Softwarelösungen. Im Rahmen dieser Studienarbeit wird Knox Premium mit der Verbindung Knox Worksapace als Lösung genauer betrachtet.



\section{MobileIron}

\subsection {Allgemein} 
Das Unternehmen MobileIron ist ein US-amerikanisches Unternehmen mit Hauptsitz in Kalifornien welches im Jahr 2007 gegründet wurde. MobileIron hat sich von Anfang an auf die Verwaltung von mobilen Endgeräten im Enterprise Umfeld spezialisiert. Das Unternhemen wurde 2017 im siebten Jahr in folge als Leader im Magic Quadrant von der Gartner Inc. neben VMWare, IBM und BlackBerry für MDM/EMM Suites gekürt. Das Softwareentwicklungsunternehmen bietet in Ihrem Produktportfolio verschiedene Bring Your Own Device Pakete mit zahlreichen Funktionen an. 

\subsection {Paketmodelle}
MobileIron bietet die drei verschiedenen Bundles „EMM Silver“, „EMM Gold“ oder „EMM Platinum“ seiner Bring Your Own Device Lösung an.
Das Basispaket „EMM Silver“ beinhaltet die Komponenten „Core“ „Sentry „und „Apps@Work“. Das Paket „EMM Gold“ ist um die Module „Email+“, „Docs@Work“ und „Web@Work“ erweitert. Durch die Wahl des Platinum Pakets ergänzt sich dieses wiederum um „Help@Work“, „Tunnel“, „MobileIron Monitor“ und „ServiceConnect-Integration“.

\includegraphics[width=0.95\textwidth]{Bilder/mi_1.png} 

\subsection {Abrechnungsmodell}

Je nach Tarifplänen bzw. Paketangeboten werden neben den genannten Grundfunktionen weitere Features unterstützt. Das Unternehmen selbst betreibt ein sehr flexibles Abrechnungsmodell, welches auf jegliche Bedürfnisse des Endkunden angepasst werden kann. Dabei kann beispielsweise zwischen einer Lizenzierung pro Benutzer (maximal 3 Endgeräte) oder einem Lizenzierungsmodell je nach Endgerät gewählt werden. Neben der Kaufoption von Lizenzen auf Lebenszeit wird auch ein Abonnement angeboten. 

\includegraphics[width=0.95\textwidth]{Bilder/mi_2.png} 
\include{Inhalt/zusammenfassung}


% ---------------------------- Literaturverzeichnis ----------------------------------------------

\printbibliography

% ------------------------------- Anhang ---------------------------------------------------------

\begin{appendix}
\clearpage
\pagenumbering{Roman}						% römische Seitenzahlen für Anhang
\end{appendix}


\end{document}
