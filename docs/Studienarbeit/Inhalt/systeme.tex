\chapter{Systeme}
\label{cha:systeme}

\section{Samsung Knox}


Ist man Besitzer aktueller Samsung-Geräten findet man die Applikation \textit{Sicherer Ordner}\footnote{Sicherer Ordner löste am 19. Dezember 2017 den Vorgänger MyKnox ab {sam2017b} }als vorinstallierte Standardsoftware vor. Mit Öffnen dieser App können, nach Eingabe eines benutzerdefinierten Sicherheitsverfahren, verschiedene Einstellungen getätigt werden. Es ist möglich Dateien oder Apps in diesen ""sicheren Ordner" zu verschieben. Sogar Apps die vorher nicht auf dem Smartphone vorhanden sind, können direkt vom Store geladen und installiert werden.Theoretisch wäre dieser Lösungsansatz genau richtig für die Verwendung von BYOD und zusätzlich sogar kostenlos. Dennoch wäre dies nicht umsetzbar im Enterprise-Umfeld.

Um den Anforderungen an eine BYOD-Lösung der Loco AG gerecht zu werden, benötigt es eine MDM-Möglichkeit. Dafür muss die IT-Administration die Möglichkeit haben die eingesetzten Geräte zu verwalten und somit an die firmeninternen Sicherheitsanforderungen anzupassen. Eine mögliche Lösung bietet Samsung mit der kostenpflichtigen Variante Samsung Knox Premium, die im Folgenden nach dem Kriterienkatalog belichtet werden soll.


Das Sicherheitsverfahren der Knox-Plattform aus fünf Komponenten \cite{sam2017}:
\begin{enumerate}
\item Mehrschichtige Sicherheit
\item Root-of-Trust
\item Secure Boot und Trusted Boot
\item TrustZone®
\item SE for Android
\end{enumerate}




