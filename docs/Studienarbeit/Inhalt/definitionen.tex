\chapter{Definitionen}
\label{cha:Definitionen}

\section{Bring Your Own Device}
Bring Your Own Device (BYOD) beschreibt eine IT-Richtlinie, die wie zu deutsch "Bring dein eigenes Gerät" beschreibt, Mitarbeitern eines Unternehmens erlaubt, eigene private Geräte im geschäftlichen Umfeld zu nutzen. Zu den Geräten dieser "Trend"-Richtlinie gehören hauptsächlich die Nutzung von Smartphones, aber auch Tablets oder Laptops sind denkbar umsetzbar. Um BYOD umsetzen zu können, werden EMM-Plattformen eingesetzt, welche die Trennung einer Arbeits- und Privatwelt auf den Geräten ermöglicht und dabei die Sicherheitsrichtlinien des Unternehmens einhält.

\subsection{Warum Bring Your Own Device?}
Bei BYOD wird häufig als Nutzer-getriebenes Thema vermarktet. Erhöhte Mitarbeiterzufriedenheit treibt als Hauptaugenmerkt und Vorteil für die meisten Umsetzungen dieser Richtlinie voran. Aber sind Unternehmen wirklich nur auf Verbesserung der Verhältnisse der Mitarbeiter aus oder gibt aus auch weitere Vorteile? Im Folgenden sollen verschiedene Vorteile von BYOD belichtet werden.





als Hilfsmittel zur Steigerung von mobiler Verfügbarkeit

\section{Mobile Device Management (MDM)}
Mobile Device Management (MDM) ist eine Technologie zur Lebenszyklusverwaltung, mit der die IT Mobilgeräte durch auf den Geräten installierte MDM-Profile einsetzen, konfigurieren, verwalten, unterstützen und sichern kann. MDM Software ermöglicht Anlageninventur, Over-the-Air-Konfiguration von E-Mail, Anwendungen und WLAN, Remotefehlerbehebung sowie Remotesperr- und Remote Wipe-Funktionen zur Sicherung von Geräten und den darauf befindlichen Unternehmensdaten. MDM ist die Grundlage für eine umfassende Enterprise Mobility Management-Lösung (EMM). 

\section{Mobile Application Management}
Mobile Application Management-Technologien (MAM) wenden Tools der Verwaltungs- und Richtlinienkontrolle auf individuelle Anwendungen statt auf das gesamte Gerät an. Üblicherweise bieten MAM-Lösungen einen benutzerdefinierten App Store, der die Kontrolle und Bereitstellung von sowohl intern entwickelten als auch Drittanbieteranwendungen erlaubt. IT-Administratoren haben die Möglichkeit, mithilfe von AppConfig Community-Standards oder Software Development Kit- oder App Wrapping-Lösungen vom MAM-Anbieter der Anwendung Sicherheits-, Verschlüsselungs- und Kontrollfunktionen hinzuzufügen. 

\section{Mobile Content Management}

\section{Enterprise Mobility Management}
Enterprise Mobility Management (EMM) ist eine geräte- und plattformagnostische Lösung, in der die Verwaltung, Konfiguration und Sicherheit aller Geräte – BYO sowie unternehmenseigenen – einer Organisation zusammengefasst werden. EMM erstreckt sich über traditionelle Geräteverwaltung hinaus auf die Verwaltung und Konfiguration von Unternehmensanwendungen und -inhalten.  Eine solide EMM-Lösung umfasst üblicherweise MDM, MAM, Mobile Content Management (MCM), Identitätsmanagement zur Zugriffskontrolle sowie Produktivitätsanwendungen zum mühelosen Zugriff auf E-Mail, Kalender, Kontakte, Inhalts-Repositorys und Intranet-Sites des Unternehmens. Darüber hinaus sollte eine EMM-Lösung es technisch ermöglichen, der IT die Verwaltung und Sicherheit zu erleichtern und gleichzeitig den Mitarbeitern ein angenehmes Benutzererlebnis zu bieten. 

\section{Unified Endpoint Management}
Mithilfe von Unified Endpoint Management (UEM) ist die IT endlich in der Lage, sich der unterschiedlichen Tools für die Verwaltung von Mobilgeräten, Desktops und seit Kurzem auch IdD-Geräten (Internet der Dinge) zu entledigen. Durch die Kombination von traditioneller Client-Verwaltung von Desktop- und PC-Systemen mit einem modernen Enterprise Mobility Management Framework (EMM) schaffen UEM-Lösungen die Voraussetzung für einen ganzheitlichen und anwenderorientierten Ansatz zur Verwaltung aller Endpunkte. Eine solide UEM-Lösung befähigt die IT zur Verwaltung von Benutzern und zur Realisierung eines einheitlichen Erlebnisses entlang aller Endpunkte sowie zur Sicherung und Verwaltung des gesamten Gerätelebenszyklus – und alles über eine zentrale, umfassende Plattform. 






