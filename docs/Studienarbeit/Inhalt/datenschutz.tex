\chapter{Datenschutz}
Durch den Einsatz privater Endgeräte im Unternehmensumfeld kann es zu wechselseitigen Zugriffsmöglichkeiten auf private, personenbezogene Daten oder besonders schützenswerte Unternehmensdaten kommen. 
Um die Interessen natürlicher Personen zu wahren existieren Rechtsvorschriften. Für dem Schutz von Unternehmensdaten hingegen müssen zusätzlich Maßnahmen zur Absicherung ergriffen werden.


\section {Rechtsrahmen}
Vor der Einführung einer Bring Your Own Device Lösung in einem Unternhemen entsteht juristischer Klärungsbedarf bezüglich der Gültigkeit verschiedener Rechtsvorschriften. 
\subsection{Bundesdatenschutzgesetz}
Das Bundesdatenschutzgesetz schreibt vor, dass 

\subsection{Europäische Datenschutz Grundverordnung}
\subsection{Arbeitsrecht}
Betriebsvereinbarungen, Vereinbarungen im Arbeitsvertrag. 
\subsection{Handelsgesetz}
\subsection{Strafrecht}
\subsection{Steuerrecht}
\section {Schützenswerte Daten}

Als schützenswerte Daten werden alle Unternehmensinternen Informationen bezeichnet, die für 



\section {Weitere Aspekte}

\subsection{Haftung}













